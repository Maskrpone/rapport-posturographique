\section{Les différentes machines}

\subsection{Plateformes de force}

Ces plateformes mesurent les forces appliquées par les pieds sur leur surface grâce à des capteurs, basés sur des jauges de contrainte ou des capteurs piézoélectriques. 
Ces capteurs enregistrent les composantes des forces dans les trois axes (X, Y, Z) et les moments de force autour de ces axes. 
Les données relevées permettent de calculer le \textbf{Centre de Pression (CdP)}, reflétant la répartition du poids et des ajustements posturaux du corps. 
Le CdP est alors suivi en temps réel dans le but d'analyser les oscillations posturales.

Elles sont fréquemment utilisées pour évaluer l’équilibre postural, la répartition du poids ainsi que les ajustements dynamiques du corps. 
Elles sont essentielles dans des domaines comme la neurologie (troubles de l’équilibre), la rééducation (suivi de récupération post-blessure), et le sport (analyse de la performance ou de l’impact). 
Ces dispositifs contribuent également à identifier les asymétries et à évaluer le risque de chute chez les personnes âgées.

\subsubsection{Exemples :}
\begin{itemize}
  \item test
\end{itemize}
