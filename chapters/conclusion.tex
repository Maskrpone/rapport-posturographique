\section{Conclusion}

La création d'une plateforme posturographique est non négligeable lorsque  l'analyse des signaux posturagraphique joue un rôle aussi essentiel dans le diagnostic médical et le suivi des patients. Face à la complexité des données reçues à l’issue des examens posturographiques, il est primordial de proposer un outil simplifiant cette analyse, tout en permettant aux médecins de gagner en efficacité et en précision. L'objectif est ainsi de rendre les résultats plus accessibles, compréhensibles et exploitables rapidement.
Au cours des quatre semaines de travail,  plusieurs étapes ont été essentielles pour concrétiser ce projet. Tout d'abord, des recherches approfondies ont été menées sur la posturographie, en particulier la posturographie statique, afin de comprendre les principes de mesure et les signaux générés par les différentes techniques. Cette étape a permis de définir clairement les besoins et les fonctionnalités attendues. Ensuite, nous avons imaginé une première version de la plateforme en élaborant une structure fonctionnelle capable de répondre aux exigences d'analyse et d'utilisation.
Les résultats obtenus jusqu'à présent témoignent d'une avancée significative dans la conceptualisation de la plateforme. Une compréhension approfondie des signaux posturographiques et des besoins des médecins a été acquise, posant ainsi des bases solides pour le développement futur.
Au cours du deuxième semestre, nous prévoyons d'orienter nos efforts vers le développement technique de la plateforme en veillant à son ergonomie ainsi que sa performance. L'objectif sera de livrer un outil fonctionnel, adapté aux besoins des médecins, capable d'offrir une analyse rapide, précise et intuitive des signaux posturographiques. Ce travail contribuera à améliorer l'efficacité des évaluations posturales et à faciliter la prise de décisions médicales.
