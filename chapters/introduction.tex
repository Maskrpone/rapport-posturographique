\section{Introduction}

Aujourd’hui, l’analyse de signaux posturographiques permet d’explorer les différents mécanismes de régulation de l’équilibre d’un patient dans différents environnements :  expérimentaux et réels. Cette analyse joue un rôle important dans divers contextes, tel que le suivi de la progression de la stabilité d’un athlète professionnel, celle d’un patient en rétablissement d’une opération chirurgicale ou encore d’une maladie neurologique. À partir de ces données, des stratégies curatives peuvent être mises en place afin d’adapter les traitements de manière individuelle.\\
 La fiabilité d’une plateforme pour l’analyse du signal posturographique est donc cruciale afin d’être en mesure d’adapter les traitements à un patient. Cette plateforme doit être intuitive, et permettre d’apporter des résultats pertinents, car elle servira d’outil d’aide à la prise de décisions pour les praticiens qui l’utiliseront. La visée de ce rapport bibliographique est d’effectuer un travail préparatoire en vue de la phase de développement de la dite plateforme.\\
 Ce rapport couvrira dans un premier temps la notion de posturographie et ses différents aspects, ainsi que le signal posturographique sur lequel l’étude portera. Il couvrira dans un second temps la stratégie de développement adoptée, qui comportera une étude de marché, c’est-à-dire, quelles solutions existent déjà. Toujours dans cette partie, le rapport décrira la conception de la plateforme, quels seront les outils utilisés pour le développement et quelles fonctionnalités la plateforme comportera à la fin de la phase de développement. Finalement, la dernière partie de ce rapport se concentrera sur la gestion de projet lors de la phase de développement, à savoir une présentation du planning établi et des outils utilisés pour le suivi de la mission.

