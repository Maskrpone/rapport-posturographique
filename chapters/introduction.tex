\section{Introduction}

Aujourd’hui, l’analyse de signaux posturographiques permet d’explorer les différents mécanismes de 
régulation de l’équilibre d’un patient, que ce soit dans des environnements expérimentaux ou réels. 
Cette analyse joue un rôle crucial dans divers contextes, tel que le suivi de la stabilité d’un athlète 
professionnel, la rééducation d’un patient après une intervention chirurgicale, ou encore dans la prise 
en charge de certaines pathologies neurologiques. Ces données permettent d’élaborer des stratégies 
curatives adaptées, en ajustant les traitements en fonction des besoins spécifiques de chaque patient.\\
Dans ce cadre, la fiabilité de la plateforme utilisée pour analyser les signaux posturographiques est 
essentielle. En effet, cette plateforme doit être à la fois intuitive et permettre de fournir des 
résultats pertinents, car elle servira d’outil d’aide à la prise de décisions pour les praticiens. 
Ce rapport bibliographique a pour objectif de réaliser un travail préparatoire en vue de la phase de 
développement de la dite plateforme.\\
Dans un premier temps, ce rapport couvrira la notion de posturographie et ses différents aspects, 
ainsi que le signal posturographique sur lequel l’étude portera. Dans un second temps, il couvrira la 
stratégie de développement adoptée, incluant une étude de marché, afin identifier les solutions 
existantes.  Cette section détaillera également la conception de la plateforme, les outils de 
développement sélectionnés, ainsi que les fonctionnalités prévues pour la version finale. Enfin, 
le rapport conclura par une présentation de la gestion de projet lors de la phase de développement, 
à savoir une présentation du planning établi et des outils utilisés pour le suivi de la mission.
